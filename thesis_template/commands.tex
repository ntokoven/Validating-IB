
\usepackage{dsfont}

\usepackage{siunitx}
\usepackage{bm}
\usepackage{nicefrac}       % compact symbols for 1/2, etc.
\usepackage{amsfonts}       % blackboard math symbols

\usepackage{gensymb}		% for \degree symbol


\DeclareMathOperator*{\argmax}{argmax}
\DeclareMathOperator*{\argmin}{argmin}

\newcommand{\lp}{\left(}
\newcommand{\rp}{\right)}
\newcommand{\lb}{\left[}
\newcommand{\rb}{\right]}

\DeclareMathOperator*{\tr}{\operatorname{Tr}}
\DeclareMathOperator*{\vect}{\operatorname{vec}}

\DeclareMathOperator*{\R}{\mathbb{R}}
\DeclareMathOperator*{\Cm}{\mathbb{C}}
\DeclareMathOperator*{\Z}{\mathbb{Z}}
\DeclareMathOperator*{\N}{\mathbb{N}}
\DeclareMathOperator*{\Sp}{S}

% Group stuff
\newcommand{\GL}[1]{\ensuremath{\operatorname{GL}(#1)}}
\newcommand{\E}[1]{\ensuremath{\operatorname{E}(#1)}}
\newcommand{\SE}[1]{\ensuremath{\operatorname{SE}(#1)}}
\renewcommand{\O}[1]{\ensuremath{\operatorname{O}(#1)}}
\newcommand{\SO}[1]{\ensuremath{\operatorname{SO}(#1)}}
\newcommand{\bE}[1]{\ensuremath{\mathbf{\operatorname{\bm E}{(#1)}}}}
\newcommand{\bO}[1]{\ensuremath{\mathbf{\operatorname{\bm O}{(#1)}}}}
\newcommand{\bSO}[1]{\ensuremath{\mathbf{\operatorname{\bm SO}(#1)}}}
\newcommand{\U}[1]{\ensuremath{\operatorname{U}(#1)}}
% discrete
\newcommand{\D}[1]{\ensuremath{\operatorname{D}_{#1}}}
\newcommand{\C}[1]{\ensuremath{\operatorname{C}_{#1}}}
\newcommand{\DN}{\ensuremath{\operatorname{D}_{\!N}}}
\newcommand{\CN}{\ensuremath{\operatorname{C}_{\!N}}}
\newcommand{\bC}[1]{\ensuremath{\mathbf{\operatorname{\bm C}_{#1}}}}
\newcommand{\bD}[1]{\ensuremath{\mathbf{\operatorname{\bm D}_{#1}}}}
\newcommand{\bCN}{\ensuremath{\mathbf{\operatorname{\bm C}_{\!N}}}}
\newcommand{\bDN}{\ensuremath{\mathbf{\operatorname{\bm D}_{\!N}}}}
\newcommand{\Flip}{(\{\pm 1\}, *)}
\newcommand{\T}[1]{\ensuremath{
    \ifthenelse{\equal{#1}{1}}
    {
        ( {\R}, + )
    }
    {
        ( {\R}^{#1}, + )
    }
}}

\newcommand{\dT}[1]{\ensuremath{
    \ifthenelse{\equal{#1}{1}}
    {
        ( {\Z}, + )
    }
    {
        ( {\Z}^{#1}, + )
    }
}}


\newcommand{\Ind}[2]{\ensuremath{\operatorname{Ind}_{#1}^{#2}}}
\newcommand{\Res}[2]{\ensuremath{\operatorname{Res}_{#1}^{#2}}}

\newcommand{\Hom}[2]{\ensuremath{\operatorname{Hom}_{#1}\lp{#2}\rp}}

\newcommand{\RCosets}[2]{\ensuremath{{#2} \backslash {#1}}}
\newcommand{\LCosets}[2]{\ensuremath{{#1} / {#2}}}

\newcommand{\Span}[1]{\ensuremath{\operatorname{Sp}\lp{#1}\rp}}

\newcommand{\vc}[1]{\ensuremath{\operatorname{vec}\!\lp{#1}\rp}}


% rotation matrices
\newcommand{\PSI}[1]{
  \begin{bmatrix}
    \cos\lp#1\rp & \!\!\!         \shortminus \sin\lp#1\rp \\
    \sin\lp#1\rp & \!\!\!\phantom{\shortminus}\cos\lp#1\rp \\
  \end{bmatrix}
}
\newcommand{\PSIP}[1]{
  \begin{bmatrix}
             \shortminus \sin\lp#1\rp & \!\!\!         \shortminus \cos\lp#1\rp \\
    \phantom{\shortminus}\cos\lp#1\rp & \!\!\!         \shortminus \sin\lp#1\rp \\
  \end{bmatrix}
}
\newcommand{\PSIS}[1]{
  \begin{bmatrix}
    \cos\lp#1\rp & \!\!\!\phantom{\shortminus}\sin\lp#1\rp \\
    \sin\lp#1\rp & \!\!\!         \shortminus \cos\lp#1\rp \\
  \end{bmatrix}
}

\newcommand{\XI}[1]{
  \begin{bmatrix}
    1 & \!\!\! 0 \\
    0 & \!\!\! #1 \\
  \end{bmatrix}
}


\DeclareMathOperator*{\EX}{\mathbb{E}}
\DeclareMathOperator*{\Cov}{Cov}
\newcommand{\X}{\mathcal{X}}
\newcommand{\Loss}{\mathcal{L}}
\newcommand{\K}{\mathcal{K}}
\newcommand{\B}{\mathcal{B}}

\newcommand{\BF}{\mathcal{BF}}

\newcommand{\muvec}{{\bm \mu}}
\newcommand{\pivec}{{\bm \pi}}
\newcommand{\thetavec}{{\bm \theta}}
\newcommand{\bvec}{{\bm b}}
\newcommand{\mvec}{{\bm m}}
\newcommand{\svec}{{\bm s}}
\newcommand{\tvec}{{\bm t}}
\newcommand{\vvec}{{\bm v}}
\newcommand{\wvec}{{\bm w}}
\newcommand{\xvec}{{\bm x}}
\newcommand{\yvec}{{\bm y}}
\newcommand{\zvec}{{\bm z}}
\newcommand{\pvec}{{\bm p}}
\newcommand{\qvec}{{\bm q}}
\newcommand{\Xvec}{{\bm X}}
\newcommand{\Yvec}{{\bm Y}}
\newcommand{\Zvec}{{\bm Z}}
\newcommand{\Sigvec}{{\bm \Sigma}}
\newcommand{\zerovec}{{\bm 0}}

\newcommand{\xtvec}{{\tilde{\xvec}}}
\newcommand{\ytvec}{{\tilde{\yvec}}}


\newcommand{\ind}{\,\rotatebox[origin=c]{90}{$\models$}\,}
\usepackage{centernot}
\newcommand{\nind}{\centernot{\rotatebox[origin=c]{90}{$\models$}}}

\usepackage{mathtools}
\DeclarePairedDelimiter{\ceil}{\lceil}{\rceil}
\DeclarePairedDelimiter{\floor}{\lfloor}{\rfloor}

\DeclareMathSymbol{\shortminus}{\mathbin}{AMSa}{"39}


% differential symbol for integrals
\newcommand{\dif}{\ensuremath{\operatorname{d}}\!}


% input and output subscripts
\newcommand{\out}{_\text{out}}
\newcommand{\inp}{_\text{in}}

% antidiagonal dots
\newcommand{\adots}{\text{\reflectbox{$\ddots$}}}


% Code snippets

\usepackage{listings}
\usepackage[numbered]{matlab-prettifier}

\definecolor{backcolour}{rgb}{0.92,0.92,0.92}

\lstdefinestyle{mymatstyle}{%
  style=Matlab-editor,
  basicstyle=\mlttfamily\footnotesize,
  backgroundcolor=\color{backcolour},
  frame=leftline,
  numberstyle=\scriptsize,
  xleftmargin=1.em,
}



\newtcbox{\kernelspace}[1][]{%
    nobeforeafter, math upper, tcbox raise base, enhanced,
    colframe=white!25!black,
    colback=white!92!black,
    boxrule=1.2pt,
    #1}



